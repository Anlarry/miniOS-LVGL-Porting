\documentclass[UTF8]{ctexart}
\usepackage[utf8]{inputenc}
\usepackage[T1]{fontenc}
\usepackage[most]{tcolorbox}
\usepackage{xcolor}

\usepackage{ctex}
\usepackage{minted}
\usepackage{graphicx}
\usepackage{pdflscape}
\usepackage{titlesec}
\usepackage{float}
\usepackage{color}
\usepackage{amsmath}
\usepackage[export]{adjustbox}
\usepackage{fancyhdr}
\usepackage{natbib}
\usepackage{subfigure}

   

\usepackage[colorlinks, 
            linkcolor=black,
            anchorcolor=black,
            citecolor=black
]{hyperref}
% \definecolor{halfgray}{gray}{0.55}

\setcounter{secnumdepth}{4}

\usepackage{geometry}
\geometry{a4paper,bottom=2cm,top=2.5cm,left=2.5cm,right=2.5cm}
\pagestyle{fancy}
\lhead{}
\rhead{\small \leftmark \ \textbf{\textcolor[rgb]{0.2,0.2,0.2}{|}} \thepage}
\renewcommand{\headrulewidth}{0pt}

\definecolor{Warning}{RGB}{255, 119, 0}
\definecolor{About\ Sigaction}{RGB}{0, 255, 204}
\definecolor{thinkcolor}{RGB}{227,196,144}

% \titleformat{\paragraph}
% {\normalfont\normalsize\bfseries}{\theparagraph}{1em}{}
% \titlespacing*{\paragraph}
% {0pt}{3.25ex plus 1ex minus .2ex}{1.5ex plus .2ex}


\begin{document}

% \newtcolorbox{mybox}[2][]
% {
%     enhanced,,sharp corners,
%     title=#2,
%     coltitle=black,
%     colback = #2, 
%     colframe = #2, 
%     colbacktitle = #2!95!black, 
%     fonttitle = \bfseries,
%     boxed title style={
%         skin=enhancedfirst jigsaw,
%         size=small,
%         arc=5mm,
%         bottom=0mm,
%         left=5mm,
%         right=18mm,
%         top=1mm},
%     attach boxed title to top left={yshift=-0mm}
% }
\newtcolorbox{mybox}[2][]
{
    tile,title=#2,colback=#2!10,colbacktitle=#2!75!black
}

% -------------------------------

\bibliographystyle{plain}

\renewcommand{\contentsname}{Contents}
\renewcommand{\bibname}{reference}

\renewcommand{\today}{\number\year-\number\month-\number\day}

\title{{\vspace{-50pt}\Huge Signal介绍及实现方法 \linebreak\linebreak}}
%please write your name, Student #, and Class # in Authors, student ID, and class # respectively
\author{}

\date{\vspace{-50pt}\today}

\maketitle

\setcounter{secnumdepth}{4}
\tableofcontents

\section{signal介绍}

信号(signal),基本思想是软件模拟硬件中断。信号,为 Linux 提供了一种处理异步事件的方法,比如终端用户输入了ctrl+c产生SIGINT来中断程序。

\section{signal的使用}

使用kill向某个进程发送信号,出现软中断,类似于硬件中断,该进程执行相应的处理函数(这里将其称为handler),完成后返回被中断的地方。下面是signalTest\_1.c中的代码

\begin{minted}[linenos,mathescape,baselinestretch=1]{cpp}
void handler(int sig, uint32_t arg) {
    printf("hanlder %d %d", sig, arg);
    for(int i = 0; i < 100; i++) {
        ;
    }
}

void main(int arg,char *argv[]) {
    int pid = get_pid();
    int sig = 0;
    int fork_res = fork();
    if(fork_res < 0) {
        printf("fork error\n");
    }
    if(fork_res == 0) {
        int tmp = get_pid();
        int tick = 0;
        while(1) {
            for(int i = 0; i < 5000000; i++) ;
            if(kill(pid, sig, tick++) < 0) {
                printf("not register");
            }
        }
    }
    else {
        for(int i = 0; i < 10000000; i++) ;
        signal(sig, handler);
        
        while(1) {
            for(int i = 0; i < 1000000; i++) ;
            printf(". ");
        }
    }
}
\end{minted}

\begin{figure}[ht]
    \centering
    \includegraphics[width = 0.8\textwidth]{fig1.png}
    \caption{signalTest}
    \label{fig:signalTest}
\end{figure}


在这个例子中,fork一个子进程,子进程调用kill不断向父进程发送0号信号,父进程调用signal注册hanlder,\autoref{fig:signalTest}。


\begin{mybox}{Warning}
    下面是signal的特点,会和linux有一些差异:
    \begin{itemize}
        \item 没有缺省的handler,如果kill在signal函数之前调用,会返回-1
        \item 这里的kill可以发送一个整型数据给hanlder。handler最多可以有两个参数,第一个是signo,第二个接收kill发送的整型数据。
        \item 当出现多个相同的信号时会合并为一个信号处理,与linux相似
        \item 这里的signal存在优先级,0号信号优先级最高
    \end{itemize}
\end{mybox}

\section{signal实现方法}

为了弄清楚signal的实现细节,首先我们需要了解signal大致流程。

\subsection{signal处理过程}

首先需要通过signal注册handler,通过kill可向其发送信号。我们从kill发送信号开始,并假设已经注册了handler。假设接收信号的进程为B,发送信号的进程为A。

\begin{figure}[ht]
    \centering
    \includegraphics[width = 0.8\textwidth]{fig3.jpg}
    \caption{linux中的signal内核处理\cite{linux-signal-zhihu}}
    \label{fig:signal_linux}
\end{figure}

A向B发送kill,陷入内核,当B再次被调到时,在B从内核返回用户态之前,处理接收到的信号,返回到handler。在handler结束后,仍需返回内核,恢复到之前被软中断打断的地方
\cite{linux-sinal-csdn}\cite{linux-signal-meizu}。可以看到存在多次进出内核。

\subsection{实现方法}

\subsubsection{思路}

A调用kill向B发送信号,会在B的PCB中置位,当调度到B时,为了让B进程从内核返回用户态后,执行handler,这个过程和系统调用密切相关,因此我们先分析系统调用内核栈的变化\autoref{fig:sys_stack}。

\begin{figure}[ht]
    \centering
    \includegraphics[width = 0.3\textwidth]{fig4.png}
    \caption{系统调用内核栈变化}
    \label{fig:sys_stack}
\end{figure}

发生系统调用时,CPU根据当前TSS中的ss0、esp0,向栈中压入eip、esp、eflag等\textbf{6}个数据\cite{mit}。当iret返回时,从此恢复。
也就是说我们需要修改这里的eip让其指向handler,让esp指向handler的上下文。

因为修改了内核栈的寄存器,handler处理结束后还需要返回到之前被中断的地方,因此需要保存原来内核中的堆栈,当handler结束后返回内核进行恢复\autoref{fig:stack_change}。

\begin{figure}[ht]
    \centering
    \subfigure[]{
        \begin{minipage}{0.4\linewidth}
            \centering
            \includegraphics[width=0.67\textwidth]{fig5.png}
        \end{minipage}
    }
    \subfigure[]{
        \begin{minipage}{0.4\linewidth}
            \centering
            \includegraphics[width=0.8\textwidth]{fig6.png}
        \end{minipage}
    }
    \caption{(a)为正常进入内核的堆栈,从esp\_save\_syscall指向的位置到ss0:esp0是我们需要保存的,(b)将原内核中的上下文保存到用户栈,并修改eip和esp。
    see more in \cite{minix-cnblog}
    }
    \label{fig:stack_change}
\end{figure}

现在的问题变成了,如何实现handler返回后回到内核进行恢复。这里的方法是用\textbf{Handler}包装handler,让其调用\mintinline{cpp}{sigreturn}陷入内核。

\subsubsection{细节}

\begin{minted}[linenos,mathescape,baselinestretch=1]{cpp}
/*
 * before B back to user space, call this to process signal,
 * construct Handler context and change esp、eip
 */
void process_signal() {
    /* speed up */
    if(!p_proc_current->task.sig_set) {
        return ;
    }

    int sig = -1, arg;
    for(int i = 0; i < 32; i++) {
        if(p_proc_current->task.sig_set & (1 << i) 
            && p_proc_current->task.sig_handler[i] /* for stable */) 
        {
            p_proc_current->task.sig_set ^= (1 << i);
            sig = i, arg = p_proc_current->task.sig_arg[i];
            break;
        }
    }
    if(sig == -1) {
        return ;
    }

    STACK_FRAME regs;
    PROCESS* proc = p_proc_current;
    memcpy(&regs, &proc->task.regs, sizeof(STACK_FRAME));
    Sigaction sigaction = {
        .sig = sig,
        .handler = proc->task.sig_handler[sig],
        .arg = proc->task.sig_arg[sig]
    };

    disable_int();

    uint16_t B_ss = regs.ss & 0xfffc;
    uint16_t A_es ;
    
    /* change es to B_ss */
    __asm__  (
        "mov %%es, %%ax\n"
        "mov %%ebx, %%es\n"
        : "=a"(A_es)
        : "b"(B_ss)
        :
    );
    
    /* save context */
    int start = *(uint32_t*)(proc->task.esp_save_syscall + 16*4) - sizeof(regs);
    for(uint32_t* p = start, *sf = proc->task.esp_save_syscall, i=0; i<sizeof(regs) / sizeof(uint32_t) ; i++,p++, sf++) {
        __asm__ ( 
            "mov %%eax, %%es:(%%edi)"
            : 
            : "a"(*sf), "D"(p)
            :
        );
    }

    /* push para */
    start -= sizeof(Sigaction);
    for(uint32_t* p = start, *sf = &sigaction, i = 0; i < sizeof(Sigaction) / sizeof(uint32_t) ;i++, p++, sf++) {
        __asm__ (
            "mov %%eax, %%es:(%%edi)"
            :
            : "a"(*sf), "D"(p)
            :
        );
    }

    /* Handler return address */
    start -= sizeof(uint32_t);

    /* switch to Handler */
    uint32_t *context_p = proc->task.esp_save_syscall;
    *(context_p + 13) =  proc->task._Hanlder; /* eip */
    *(context_p + 16) = start; /* esp */

    /*  reverse  */
    __asm__ __volatile__ (
        "mov %0, %%es\n"
        : 
        : "r"(A_es)
        :
    );
    enable_int();
}

\end{minted}

在上面的函数中,处理用户堆栈分为:保存原kernel stack中的寄存器值、构造包装函数\textbf{Handler}的参数等。对kernel栈中的变化为:修改eip、esp。
在kernel.asm中可看到在返回用户栈之前调用该函数。

B返回用户态后执行包装函数,调用handler。

\begin{minted}[linenos,mathescape,baselinestretch=1]{cpp}
/* In the use of signal mechanism, the existence of Handler will not be felt  */
void Handler(Sigaction sigaction) {
    void (*_fun)(int, int);
    _fun = (void (*)(int, int))sigaction.handler;
    _fun(sigaction.sig, sigaction.arg);
    int ebp;
    __asm__ __volatile__ (
        "mov %%ebp, %0"
        : "=r"(ebp)
        : 
        :
    );
    sigreturn(ebp); /* trap into kernel*/
}
\end{minted}

\begin{mybox}{About\ Sigaction}
    这里的Sigaction是关于signal的信息一个结构体。与其他操作系统定义并不完全相同。
    \begin{minted}[mathescape,baselinestretch=1]{cpp}
    typedef struct Sigaction {
        int sig; /* 信号 */
        void *handler; /* handler函数指针 */
        uint32_t arg; /* 传给handler的参数 */
    }Sigaction;
    \end{minted}

\end{mybox}

\begin{minted}[linenos,mathescape,baselinestretch=1]{cpp}
void do_sigreturn(int ebp)
{
    STACK_FRAME regs;

    // copy saved regs from stack to  this regs
    // to some operation to compute true address
    int esp_syscall = p_proc_current->task.esp_save_syscall;
    int last_esp = ebp + sizeof(Sigaction) + 8;    //int save esp

    uint16_t user_ss = p_proc_current->task.regs.ss;
    uint16_t kernel_es ;

    /* change es to B_ss */
    __asm__  (
        "mov %%es, %%ax\n"
        "mov %%ebx, %%es\n"
        : "=a"(kernel_es)
        : "b"(user_ss)
        :
    );

    for(uint32_t* p = &regs, *q = last_esp, i = 0; i < sizeof(regs) / sizeof(uint32_t);i++, p++, q++)
    {
        __asm__ (
        "mov %%eax, %%es:(%%edi)"
        :
        : "a"(*q), "D"(p)
        :
        );
    }

    __asm__ (
        "mov %%eax, %%es\n"
        :
        : "a"(kernel_es)
    );
    memcpy(esp_syscall, &regs, sizeof(STACK_FRAME));
}

\end{minted}

\mintinline{cpp}{sigreturn}中通过传入的Handler的ebp确定原kernel栈的寄存器保存位置,并恢复kernel栈中保存的寄存器。重新返回用户态后,整个信号过程就好像没发生过一样。

\subsection{多个信号的处理}

当存在多个信号时,一次处理一个信号。比如存在0号和5号信号,当0号信号处理结束后,sigreturn陷入内核、恢复山下文,在返回用户态之前的,process\_signal发现还有5号信号,再构造堆栈,返回用户态后执行5号信号的handler,sigreturn再次返回内核...。

\section{系统调用}

主要提供了三个用户接口,都对应一个系统调用。

\paragraph{signal} 注册处理函数,并且将该进程对应的Handler函数指针告诉内核。

\begin{minted}[linenos,mathescape,baselinestretch=1]{cpp}
/* user space*/
int signal(int sig, void* handler) {
    return __signal(sig, handler, HANDLER /* Handler */); /* syscall */
}
// -----------------------------------------------------
/* kernel space */
int do_signal(int sig, void *handler, void* _Handler) {
    if((uint32_t)sig >= NR_SIGNALS) {
        return -1;
    }
    p_proc_current->task.sig_handler[sig] = handler;
    p_proc_current->task._Hanlder = _Handler;
    return 0;
}
\end{minted}


\paragraph{kill} 向一个进程发送信号。在对应PCB中的sig\_set,相应位置位。并对相应的handler检查是否为空.

\begin{minted}[linenos,mathescape,baselinestretch=1]{cpp}
/* user space*/
int kill(int pid, int sig, ...) {
    va_list ap = (va_list)((char*)(&sig) + 4);
    Sigaction sigaction = {
        .sig = sig,
        .handler = NULL,
        
        .arg = *((uint32_t*)ap)
    };
    return sigsend(pid, &sigaction); /* syscall */
}
// -----------------------------------------------------
/* kernel space */
int do_sigsend(int pid, Sigaction* sigaction_p)
{
    PROCESS* proc = &proc_table[pid];
    if(proc->task.sig_handler[sigaction_p->sig] == NULL ||
        (proc->task.sig_set & (1 << sigaction_p->sig))) 
    {
        return -1;
    }
    proc->task.sig_set |= (1 << sigaction_p->sig);
    proc->task.sig_arg[sigaction_p->sig] = sigaction_p->arg;

    return 0;
}
\end{minted}

\begin{mybox}{Warning}
    在kill中并不判断是否传递了第三个参数,当没有传递第三个参数,sigaction中的arg可能会是一个较随机的值。
    因此如果handler需要使用第二个参数时,kill需要传递第三个参数。
\end{mybox}

\paragraph{sigreturn} 从signal handler恢复到原来执行的位置。\textbf{用户不应该直接调用这个系统调用}。



\bibliography{ref}

\end{document}
